% Options for packages loaded elsewhere
\PassOptionsToPackage{unicode}{hyperref}
\PassOptionsToPackage{hyphens}{url}
%
\documentclass[
]{article}
\usepackage{lmodern}
\usepackage{amssymb,amsmath}
\usepackage{ifxetex,ifluatex}
\ifnum 0\ifxetex 1\fi\ifluatex 1\fi=0 % if pdftex
  \usepackage[T1]{fontenc}
  \usepackage[utf8]{inputenc}
  \usepackage{textcomp} % provide euro and other symbols
\else % if luatex or xetex
  \usepackage{unicode-math}
  \defaultfontfeatures{Scale=MatchLowercase}
  \defaultfontfeatures[\rmfamily]{Ligatures=TeX,Scale=1}
\fi
% Use upquote if available, for straight quotes in verbatim environments
\IfFileExists{upquote.sty}{\usepackage{upquote}}{}
\IfFileExists{microtype.sty}{% use microtype if available
  \usepackage[]{microtype}
  \UseMicrotypeSet[protrusion]{basicmath} % disable protrusion for tt fonts
}{}
\makeatletter
\@ifundefined{KOMAClassName}{% if non-KOMA class
  \IfFileExists{parskip.sty}{%
    \usepackage{parskip}
  }{% else
    \setlength{\parindent}{0pt}
    \setlength{\parskip}{6pt plus 2pt minus 1pt}}
}{% if KOMA class
  \KOMAoptions{parskip=half}}
\makeatother
\usepackage{xcolor}
\IfFileExists{xurl.sty}{\usepackage{xurl}}{} % add URL line breaks if available
\IfFileExists{bookmark.sty}{\usepackage{bookmark}}{\usepackage{hyperref}}
\hypersetup{
  pdftitle={Lab0},
  pdfauthor={Jas Sur},
  hidelinks,
  pdfcreator={LaTeX via pandoc}}
\urlstyle{same} % disable monospaced font for URLs
\usepackage[margin=1in]{geometry}
\usepackage{color}
\usepackage{fancyvrb}
\newcommand{\VerbBar}{|}
\newcommand{\VERB}{\Verb[commandchars=\\\{\}]}
\DefineVerbatimEnvironment{Highlighting}{Verbatim}{commandchars=\\\{\}}
% Add ',fontsize=\small' for more characters per line
\usepackage{framed}
\definecolor{shadecolor}{RGB}{248,248,248}
\newenvironment{Shaded}{\begin{snugshade}}{\end{snugshade}}
\newcommand{\AlertTok}[1]{\textcolor[rgb]{0.94,0.16,0.16}{#1}}
\newcommand{\AnnotationTok}[1]{\textcolor[rgb]{0.56,0.35,0.01}{\textbf{\textit{#1}}}}
\newcommand{\AttributeTok}[1]{\textcolor[rgb]{0.77,0.63,0.00}{#1}}
\newcommand{\BaseNTok}[1]{\textcolor[rgb]{0.00,0.00,0.81}{#1}}
\newcommand{\BuiltInTok}[1]{#1}
\newcommand{\CharTok}[1]{\textcolor[rgb]{0.31,0.60,0.02}{#1}}
\newcommand{\CommentTok}[1]{\textcolor[rgb]{0.56,0.35,0.01}{\textit{#1}}}
\newcommand{\CommentVarTok}[1]{\textcolor[rgb]{0.56,0.35,0.01}{\textbf{\textit{#1}}}}
\newcommand{\ConstantTok}[1]{\textcolor[rgb]{0.00,0.00,0.00}{#1}}
\newcommand{\ControlFlowTok}[1]{\textcolor[rgb]{0.13,0.29,0.53}{\textbf{#1}}}
\newcommand{\DataTypeTok}[1]{\textcolor[rgb]{0.13,0.29,0.53}{#1}}
\newcommand{\DecValTok}[1]{\textcolor[rgb]{0.00,0.00,0.81}{#1}}
\newcommand{\DocumentationTok}[1]{\textcolor[rgb]{0.56,0.35,0.01}{\textbf{\textit{#1}}}}
\newcommand{\ErrorTok}[1]{\textcolor[rgb]{0.64,0.00,0.00}{\textbf{#1}}}
\newcommand{\ExtensionTok}[1]{#1}
\newcommand{\FloatTok}[1]{\textcolor[rgb]{0.00,0.00,0.81}{#1}}
\newcommand{\FunctionTok}[1]{\textcolor[rgb]{0.00,0.00,0.00}{#1}}
\newcommand{\ImportTok}[1]{#1}
\newcommand{\InformationTok}[1]{\textcolor[rgb]{0.56,0.35,0.01}{\textbf{\textit{#1}}}}
\newcommand{\KeywordTok}[1]{\textcolor[rgb]{0.13,0.29,0.53}{\textbf{#1}}}
\newcommand{\NormalTok}[1]{#1}
\newcommand{\OperatorTok}[1]{\textcolor[rgb]{0.81,0.36,0.00}{\textbf{#1}}}
\newcommand{\OtherTok}[1]{\textcolor[rgb]{0.56,0.35,0.01}{#1}}
\newcommand{\PreprocessorTok}[1]{\textcolor[rgb]{0.56,0.35,0.01}{\textit{#1}}}
\newcommand{\RegionMarkerTok}[1]{#1}
\newcommand{\SpecialCharTok}[1]{\textcolor[rgb]{0.00,0.00,0.00}{#1}}
\newcommand{\SpecialStringTok}[1]{\textcolor[rgb]{0.31,0.60,0.02}{#1}}
\newcommand{\StringTok}[1]{\textcolor[rgb]{0.31,0.60,0.02}{#1}}
\newcommand{\VariableTok}[1]{\textcolor[rgb]{0.00,0.00,0.00}{#1}}
\newcommand{\VerbatimStringTok}[1]{\textcolor[rgb]{0.31,0.60,0.02}{#1}}
\newcommand{\WarningTok}[1]{\textcolor[rgb]{0.56,0.35,0.01}{\textbf{\textit{#1}}}}
\usepackage{graphicx,grffile}
\makeatletter
\def\maxwidth{\ifdim\Gin@nat@width>\linewidth\linewidth\else\Gin@nat@width\fi}
\def\maxheight{\ifdim\Gin@nat@height>\textheight\textheight\else\Gin@nat@height\fi}
\makeatother
% Scale images if necessary, so that they will not overflow the page
% margins by default, and it is still possible to overwrite the defaults
% using explicit options in \includegraphics[width, height, ...]{}
\setkeys{Gin}{width=\maxwidth,height=\maxheight,keepaspectratio}
% Set default figure placement to htbp
\makeatletter
\def\fps@figure{htbp}
\makeatother
\setlength{\emergencystretch}{3em} % prevent overfull lines
\providecommand{\tightlist}{%
  \setlength{\itemsep}{0pt}\setlength{\parskip}{0pt}}
\setcounter{secnumdepth}{-\maxdimen} % remove section numbering

\title{Lab0}
\author{Jas Sur}
\date{1/26/2022}

\begin{document}
\maketitle

\hypertarget{lab-0-problems}{%
\section{Lab 0 Problems}\label{lab-0-problems}}

\hypertarget{question-1.a}{%
\subsection{Question 1.a}\label{question-1.a}}

\hypertarget{generate-a-random-sample-of-100-from-t-distribtion-degree-of-freedom-10.}{%
\subsubsection{Generate a random sample of 100 from t-distribtion,
degree of freedom
10.}\label{generate-a-random-sample-of-100-from-t-distribtion-degree-of-freedom-10.}}

\hypertarget{check-qqnorm-qqline-shapiro-test.-remarks}{%
\subsubsection{Check qqnorm(); qqline(), Shapiro test.
Remarks}\label{check-qqnorm-qqline-shapiro-test.-remarks}}

\begin{Shaded}
\begin{Highlighting}[]
\NormalTok{sample.t <-}\StringTok{ }\KeywordTok{rt}\NormalTok{(}\DecValTok{100}\NormalTok{, }\DataTypeTok{df=}\DecValTok{10}\NormalTok{)}
\KeywordTok{qqnorm}\NormalTok{(sample.t)}
\KeywordTok{qqline}\NormalTok{(sample.t)}
\end{Highlighting}
\end{Shaded}

\includegraphics{lab0_files/figure-latex/q1.b-1.pdf}

\begin{Shaded}
\begin{Highlighting}[]
\KeywordTok{shapiro.test}\NormalTok{(sample.t)}
\end{Highlighting}
\end{Shaded}

\begin{verbatim}
## 
##  Shapiro-Wilk normality test
## 
## data:  sample.t
## W = 0.98574, p-value = 0.3588
\end{verbatim}

\begin{center}\rule{0.5\linewidth}{0.5pt}\end{center}

\begin{enumerate}
\def\labelenumi{\arabic{enumi}.}
\tightlist
\item
  Null hypothesis (H0 sample data is normal)
\item
  Alternative hypothesis (H1 sample data is normal)
\end{enumerate}

Since the pvalue \textgreater{} 0.05, which means that the null
hypothesis that values are normal is accepted and the alternative
hypothesis is rejected. ***

\hypertarget{question-1.b}{%
\subsection{Question 1.b}\label{question-1.b}}

\hypertarget{generate-a-random-sample-of-100-from-a-chi-square-distribution-df-5.}{%
\subsubsection{Generate a random sample of 100 from a Chi-square
distribution, df
=5.}\label{generate-a-random-sample-of-100-from-a-chi-square-distribution-df-5.}}

\hypertarget{perform-same-procedures-as-in-a-.-remarks.}{%
\subsubsection{Perform same procedures as in (a) .
Remarks.}\label{perform-same-procedures-as-in-a-.-remarks.}}

\begin{Shaded}
\begin{Highlighting}[]
\NormalTok{sample}\FloatTok{.1}\NormalTok{b <-}\KeywordTok{seq}\NormalTok{(}\OperatorTok{-}\DecValTok{4}\NormalTok{, }\DecValTok{4}\NormalTok{, }\FloatTok{0.01}\NormalTok{)}
\NormalTok{sample.chi =}\StringTok{ }\KeywordTok{dchisq}\NormalTok{(sample}\FloatTok{.1}\NormalTok{b, }\DataTypeTok{df=}\DecValTok{5}\NormalTok{)}
\KeywordTok{qqnorm}\NormalTok{(sample.chi)}
\KeywordTok{qqline}\NormalTok{(sample.chi)}
\end{Highlighting}
\end{Shaded}

\includegraphics{lab0_files/figure-latex/q.1b-1.pdf}

\begin{Shaded}
\begin{Highlighting}[]
\KeywordTok{shapiro.test}\NormalTok{(sample.chi)}
\end{Highlighting}
\end{Shaded}

\begin{verbatim}
## 
##  Shapiro-Wilk normality test
## 
## data:  sample.chi
## W = 0.728, p-value < 2.2e-16
\end{verbatim}

\begin{center}\rule{0.5\linewidth}{0.5pt}\end{center}

\begin{enumerate}
\def\labelenumi{\arabic{enumi}.}
\item
  Null hypothesis (H0 sample data is normal)
\item
  Alternative hypothesis (H1 sample data is normal)
\end{enumerate}

Since the pvalue \textless{} 0.05, which means that the null hypothesis
that values are normal is rejected and the alternative hypothesis is
accepted. ***

\hypertarget{question-2.a}{%
\subsection{Question 2.a}\label{question-2.a}}

\hypertarget{write-a-95-ci-for-the-population-mean.}{%
\subsubsection{Write a 95\%-CI for the population
mean.}\label{write-a-95-ci-for-the-population-mean.}}

\hypertarget{what-assumption-about-population-for-the-work-suppose-the-sample-is-random.}{%
\subsubsection{What assumption about population for the work, suppose
the sample is
random.}\label{what-assumption-about-population-for-the-work-suppose-the-sample-is-random.}}

\begin{Shaded}
\begin{Highlighting}[]
\NormalTok{sample <-}\KeywordTok{c}\NormalTok{(}\FloatTok{26.4}\NormalTok{,}\FloatTok{23.5}\NormalTok{,}\FloatTok{25.4}\NormalTok{,}\FloatTok{22.9}\NormalTok{,}\FloatTok{25.2}\NormalTok{,}\FloatTok{39.2}\NormalTok{,}\FloatTok{25.5}\NormalTok{,}\FloatTok{31.9}\NormalTok{,}\FloatTok{26.0}\NormalTok{,}\FloatTok{44.6}\NormalTok{,}\FloatTok{35.5}\NormalTok{,}\FloatTok{38.6}\NormalTok{,}
           \FloatTok{30.1}\NormalTok{,}\FloatTok{31.0}\NormalTok{,}\FloatTok{30.8}\NormalTok{,}\FloatTok{32.8}\NormalTok{,}\FloatTok{47.7}\NormalTok{,}\FloatTok{39.1}\NormalTok{,}\FloatTok{55.3}\NormalTok{,}\FloatTok{50.7}\NormalTok{,}\FloatTok{73.8}\NormalTok{,}\FloatTok{71.1}\NormalTok{,}\FloatTok{68.4}\NormalTok{,}\FloatTok{77.1}\NormalTok{,}
           \FloatTok{19.4}\NormalTok{,}\FloatTok{19.3}\NormalTok{,}\FloatTok{18.7}\NormalTok{,}\FloatTok{19.0}\NormalTok{,}\FloatTok{23.2}\NormalTok{,}\FloatTok{21.3}\NormalTok{,}\FloatTok{23.2}\NormalTok{,}\FloatTok{19.9}\NormalTok{,}\FloatTok{18.9}\NormalTok{,}\FloatTok{19.8}\NormalTok{,}\FloatTok{19.6}\NormalTok{,}\FloatTok{21.9}\NormalTok{)}
\KeywordTok{hist}\NormalTok{(sample)}
\end{Highlighting}
\end{Shaded}

\includegraphics{lab0_files/figure-latex/q.2a-1.pdf}

\begin{Shaded}
\begin{Highlighting}[]
\NormalTok{n<-}\KeywordTok{length}\NormalTok{(sample)}
\NormalTok{xbar<-}\KeywordTok{mean}\NormalTok{(sample)}
\NormalTok{s<-}\KeywordTok{sd}\NormalTok{(sample)}
\KeywordTok{sprintf}\NormalTok{(}\StringTok{"mean %f, std dev %f, sample size %d"}\NormalTok{, xbar, s, n)}
\end{Highlighting}
\end{Shaded}

\begin{verbatim}
## [1] "mean 33.800000, std dev 16.897489, sample size 36"
\end{verbatim}

\begin{Shaded}
\begin{Highlighting}[]
\NormalTok{margin <-}\StringTok{ }\KeywordTok{qt}\NormalTok{(}\FloatTok{0.95}\NormalTok{,}\DataTypeTok{df=}\NormalTok{n}\DecValTok{-1}\NormalTok{)}\OperatorTok{*}\NormalTok{s}\OperatorTok{/}\KeywordTok{sqrt}\NormalTok{(n)}
\NormalTok{low_ci <-}\StringTok{ }\NormalTok{xbar }\OperatorTok{-}\StringTok{ }\NormalTok{margin}
\NormalTok{low_ci}
\end{Highlighting}
\end{Shaded}

\begin{verbatim}
## [1] 29.04174
\end{verbatim}

\begin{Shaded}
\begin{Highlighting}[]
\NormalTok{high_ci <-}\StringTok{ }\NormalTok{xbar }\OperatorTok{+}\StringTok{ }\NormalTok{margin}
\KeywordTok{sprintf}\NormalTok{(}\StringTok{"confidence interval for the population mean is (%f, %f)"}\NormalTok{, low_ci, high_ci)}
\end{Highlighting}
\end{Shaded}

\begin{verbatim}
## [1] "confidence interval for the population mean is (29.041745, 38.558255)"
\end{verbatim}

\begin{center}\rule{0.5\linewidth}{0.5pt}\end{center}

95 \% Confidence interval is reasonable compare to the sample mean ***

\hypertarget{question-2.b}{%
\subsection{Question 2.b}\label{question-2.b}}

\hypertarget{write-a-95--ci-for-population-standard-deviation.}{%
\subsubsection{Write a 95\%- CI for population standard
deviation.}\label{write-a-95--ci-for-population-standard-deviation.}}

confidence interval:
\(CI = [√(n-1)s^{2}/X^{2}(α/2), √(n-1)s^{2}/ X^{2}(1-α/2)]\)

\begin{Shaded}
\begin{Highlighting}[]
\NormalTok{sample <-}\KeywordTok{c}\NormalTok{(}\FloatTok{26.4}\NormalTok{,}\FloatTok{23.5}\NormalTok{,}\FloatTok{25.4}\NormalTok{,}\FloatTok{22.9}\NormalTok{,}\FloatTok{25.2}\NormalTok{,}\FloatTok{39.2}\NormalTok{,}\FloatTok{25.5}\NormalTok{,}\FloatTok{31.9}\NormalTok{,}\FloatTok{26.0}\NormalTok{,}\FloatTok{44.6}\NormalTok{,}\FloatTok{35.5}\NormalTok{,}\FloatTok{38.6}\NormalTok{,}
           \FloatTok{30.1}\NormalTok{,}\FloatTok{31.0}\NormalTok{,}\FloatTok{30.8}\NormalTok{,}\FloatTok{32.8}\NormalTok{,}\FloatTok{47.7}\NormalTok{,}\FloatTok{39.1}\NormalTok{,}\FloatTok{55.3}\NormalTok{,}\FloatTok{50.7}\NormalTok{,}\FloatTok{73.8}\NormalTok{,}\FloatTok{71.1}\NormalTok{,}\FloatTok{68.4}\NormalTok{,}\FloatTok{77.1}\NormalTok{,}
           \FloatTok{19.4}\NormalTok{,}\FloatTok{19.3}\NormalTok{,}\FloatTok{18.7}\NormalTok{,}\FloatTok{19.0}\NormalTok{,}\FloatTok{23.2}\NormalTok{,}\FloatTok{21.3}\NormalTok{,}\FloatTok{23.2}\NormalTok{,}\FloatTok{19.9}\NormalTok{,}\FloatTok{18.9}\NormalTok{,}\FloatTok{19.8}\NormalTok{,}\FloatTok{19.6}\NormalTok{,}\FloatTok{21.9}\NormalTok{)}
\NormalTok{n <-}\StringTok{ }\KeywordTok{length}\NormalTok{(sample)}
\NormalTok{xbar<-}\KeywordTok{mean}\NormalTok{(sample)}
\NormalTok{s<-}\KeywordTok{sd}\NormalTok{(sample)}
\KeywordTok{sprintf}\NormalTok{(}\StringTok{"population sd %f, size %d, mean %f"}\NormalTok{, s, n, xbar)}
\end{Highlighting}
\end{Shaded}

\begin{verbatim}
## [1] "population sd 16.897489, size 36, mean 33.800000"
\end{verbatim}

\begin{Shaded}
\begin{Highlighting}[]
\NormalTok{left <-}\StringTok{ }\KeywordTok{qchisq}\NormalTok{(}\DataTypeTok{p=}\NormalTok{.}\DecValTok{05}\NormalTok{, }\DataTypeTok{df=}\NormalTok{n}\DecValTok{-1}\NormalTok{, }\DataTypeTok{lower.tail=}\OtherTok{FALSE}\NormalTok{)}
\NormalTok{right <-}\StringTok{ }\KeywordTok{qchisq}\NormalTok{(}\DataTypeTok{p=}\NormalTok{.}\DecValTok{95}\NormalTok{, }\DataTypeTok{df=}\NormalTok{n}\DecValTok{-1}\NormalTok{, }\DataTypeTok{lower.tail=}\OtherTok{FALSE}\NormalTok{)}
\NormalTok{ci_left <-}\StringTok{ }\KeywordTok{sqrt}\NormalTok{((n}\DecValTok{-1}\NormalTok{)}\OperatorTok{*}\NormalTok{s}\OperatorTok{^}\DecValTok{2}\OperatorTok{/}\NormalTok{left)}
\NormalTok{ci_right <-}\StringTok{ }\KeywordTok{sqrt}\NormalTok{((n}\DecValTok{-1}\NormalTok{)}\OperatorTok{*}\NormalTok{s}\OperatorTok{^}\DecValTok{2}\OperatorTok{/}\NormalTok{right)}

\KeywordTok{sprintf}\NormalTok{(}\StringTok{"Confidence interval for population std dev is  %f, %f"}\NormalTok{, ci_left, ci_right)}
\end{Highlighting}
\end{Shaded}

\begin{verbatim}
## [1] "Confidence interval for population std dev is  14.165551, 21.091275"
\end{verbatim}

\begin{center}\rule{0.5\linewidth}{0.5pt}\end{center}

A confidence interval for a standard deviation is a range of values that
is likely to contain true population standard deviation with a certain
level of confidence. ***

\hypertarget{question-3}{%
\section{Question 3}\label{question-3}}

\hypertarget{run-the-following-code}{%
\subsubsection{Run the following code}\label{run-the-following-code}}

\begin{Shaded}
\begin{Highlighting}[]
\NormalTok{sample <-}\KeywordTok{c}\NormalTok{(}\FloatTok{26.4}\NormalTok{,}\FloatTok{23.5}\NormalTok{,}\FloatTok{25.4}\NormalTok{,}\FloatTok{22.9}\NormalTok{,}\FloatTok{25.2}\NormalTok{,}\FloatTok{39.2}\NormalTok{,}\FloatTok{25.5}\NormalTok{,}\FloatTok{31.9}\NormalTok{,}\FloatTok{26.0}\NormalTok{,}\FloatTok{44.6}\NormalTok{,}\FloatTok{35.5}\NormalTok{,}\FloatTok{38.6}\NormalTok{,}
           \FloatTok{30.1}\NormalTok{,}\FloatTok{31.0}\NormalTok{,}\FloatTok{30.8}\NormalTok{,}\FloatTok{32.8}\NormalTok{,}\FloatTok{47.7}\NormalTok{,}\FloatTok{39.1}\NormalTok{,}\FloatTok{55.3}\NormalTok{,}\FloatTok{50.7}\NormalTok{,}\FloatTok{73.8}\NormalTok{,}\FloatTok{71.1}\NormalTok{,}\FloatTok{68.4}\NormalTok{,}\FloatTok{77.1}\NormalTok{,}
           \FloatTok{19.4}\NormalTok{,}\FloatTok{19.3}\NormalTok{,}\FloatTok{18.7}\NormalTok{,}\FloatTok{19.0}\NormalTok{,}\FloatTok{23.2}\NormalTok{,}\FloatTok{21.3}\NormalTok{,}\FloatTok{23.2}\NormalTok{,}\FloatTok{19.9}\NormalTok{,}\FloatTok{18.9}\NormalTok{,}\FloatTok{19.8}\NormalTok{,}\FloatTok{19.6}\NormalTok{,}\FloatTok{21.9}\NormalTok{)}
\KeywordTok{hist}\NormalTok{(sample)}
\end{Highlighting}
\end{Shaded}

\includegraphics{lab0_files/figure-latex/q.3-1.pdf}

\begin{Shaded}
\begin{Highlighting}[]
\NormalTok{sample.s <-}\KeywordTok{sort}\NormalTok{(sample) }\CommentTok{#sort data increasing}
\NormalTok{rank <-}\StringTok{ }\KeywordTok{rank}\NormalTok{(sample.s) }\CommentTok{#rank data from 1 to 36}
\NormalTok{size <-}\StringTok{ }\KeywordTok{length}\NormalTok{(sample.s) }\CommentTok{# size of data}
\NormalTok{p <-}\StringTok{ }\NormalTok{(rank}\FloatTok{-.5}\NormalTok{)}\OperatorTok{/}\NormalTok{size }\CommentTok{#cummulative prob of data}
\NormalTok{z.quantile <-}\StringTok{ }\KeywordTok{qnorm}\NormalTok{(p) }\CommentTok{# Standard Normal quantiles with such probability}
\KeywordTok{plot}\NormalTok{(}\DataTypeTok{x=}\NormalTok{z.quantile, }\DataTypeTok{y=}\NormalTok{sample.s, }\DataTypeTok{pch=}\DecValTok{16}\NormalTok{, }\DataTypeTok{main=}\StringTok{"QQ Plot"}\NormalTok{) }\CommentTok{#scatterplot of x=Z quantiles, y= data sorted}
\KeywordTok{abline}\NormalTok{(}\KeywordTok{lm}\NormalTok{(sample.s }\OperatorTok{~}\StringTok{ }\NormalTok{z.quantile))}
\end{Highlighting}
\end{Shaded}

\includegraphics{lab0_files/figure-latex/q.3-2.pdf}

The QQ plot suggests that the data is not normal

\end{document}
